\documentclass[10pt,a4paper]{article}
\usepackage[utf8]{inputenc}
\usepackage[italian]{babel}
\usepackage{graphicx}
\usepackage{floatflt,epsfig}
\usepackage{graphicx}
\usepackage{pgfplots}
\usepackage{booktabs,xltabular}
\usepackage{amsmath}
\usepackage{tikz}
\usepackage{subcaption}
\usepackage{caption}
\usepackage{tabularx}
\usepackage{hyperref}
\usepackage{xcolor}
\hypersetup{
	colorlinks,
	linkcolor={red!50!black},
	citecolor={blue!50!black},
	urlcolor={blue!80!black}
}
\usepackage{pgfplots,pgfplotstable}
\usepackage{wrapfig}
\usepackage{float}
\usepackage{sidecap}
\bibliographystyle{plain}
\usepackage[export]{adjustbox}
\usepackage{xcolor}
\usepackage{listings}
\usepackage{amssymb}
\usepackage{amsmath,amssymb}
\usepackage{amscd} 
\usepackage{amsthm}
\usepackage{amssymb}
\usepackage{mathtools}

\lstset{language=Python,
	basicstyle=\small,
	keywordstyle=\color{black}\bfseries,
	commentstyle=\color{orange},
	numbers=left, numberstyle=\tiny, stepnumber=1, numbersep=5pt,
	frame=single}

\title{Modello di ising 2D e transizione di fase}
\author{Francesco Anna Mele, Marco Eterno, Edoardo Maria Centamori}

\begin{document}
	\begin{titlepage}
		\begin{center}
			
			\vspace{1cm}
			{\LARGE\textbf{ Modello di ising quantistico simulato con DMRG}\par}
			
			\vspace{0.5cm}
			{\Large Edoardo Centamori, Marco Eterno, \newline Francesco Anna Mele\par}
			
			\vspace{1cm}
			
			\includegraphics[width=0.4\linewidth]{unipi}
			
			
		\end{center}
	\tableofcontents
	\end{titlepage}
	
	\newpage
	\section{Introduzione}
	L'obiettivo dell'esperienza è simulare un modello di ising quantistico 1D, misurando delle osservabili di interesse, al variare dei parametri del problema, ovvero il campo magnetico esterno e la taglia del sistema. Il tutto è implementato  attraverso l'infinite system DMRG. 

	
	\section{Modello di Ising quantistico 1-D}
	La più generale Hamiltoniana per un sistema di spin quantistici su reticoli 1-D risulta essere
	\begin{equation}
		H=-\sum_{j} \sigma_j^z\sigma_{j+1}^z-h\sum_{j}\sigma_j^z-g\sum_{j}\sigma_j^x
		\label{hamiltonian}
	\end{equation}
	il motivo per cui non serve considerare anche un eventuale termine $-j\sum_{j}\sigma_j^y$ è che prendiamo l'asse x orientato con la componente del campo magnetico esterno che non è diretta lungo z. Si è scelto x come asse preferenziale poichè così l'hamiltoniana è reale, e questo abbassa notevolmente i tempi di computazione. 
	

	
	
	
	
	
	
	
\end{document}
